\documentclass[11pt,letterpaper]{article}
\usepackage[utf8]{inputenc}
\usepackage{graphicx}
\usepackage{tabularx}
\usepackage{multirow}
\usepackage{float}
%\decimalpoint

\begin{document}
\begin{center}
{\Large Herramientas computacionales} \\
S8CASA - \textsc{Makefile}\\
\end{center}


\noindent
\section{Gr\'aficas}
\begin{center}
\includegraphics[width=10cm]{euler.pdf}
\begin{center}
\end{center}
\includegraphics[width=10cm]{leapfrog.pdf}
\end{center}
\begin{center}
\end{center}
\includegraphics[width=10cm]{rungekutta.pdf}
\begin{center}
\end{center}
\includegraphics[width=10cm]{eulera.pdf}
\begin{center}
\end{center}
\includegraphics[width=10cm]{rungekuttaam.pdf}

\text{Viendo las gráficas, nos damos cuenta de que ell método de Euler es el más impreciso, mientras que el de Leapfrog y Runge Kutta son más estables.
En cuanto al Runge Kutta amortiguado, se nota cómo la oscilación disminuye a medida que pasa el tiempo.}

\begin{center}
\includegraphics[width=10cm]{EulerOrbit.pdf}
\begin{center}
\end{center}
\includegraphics[width=10cm]{LeapfrogOrbit.pdf}
\end{center}

\text{Para el ejercicio de la órbita de la Tierra alrededor del Sol (caso 1), la diferencia que alcanzo a percibir entre el método de Euler y el de Leapfrog, es que la órbita con Euler se ve ligeramente más imprecisa que la de Leapfrog a medida que pasan los años.}



\end{document}
